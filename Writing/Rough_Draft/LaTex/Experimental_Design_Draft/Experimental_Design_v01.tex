% AER-Article.tex for AEA last revised 22 June 2011
\documentclass[AER]{AEA}
\usepackage{caption}
\usepackage{subcaption}
\usepackage{graphicx}
\graphicspath{ {images/} }
% The mathtime package uses a Times font instead of Computer Modern.
% Uncomment the line below if you wish to use the mathtime package:
%\usepackage[cmbold]{mathtime}
% Note that miktex, by default, configures the mathtime package to use commercial fonts
% which you may not have. If you would like to use mathtime but you are seeing error
% messages about missing fonts (mtex.pfb, mtsy.pfb, or rmtmi.pfb) then please see
% the technical support document at http://www.aeaweb.org/templates/technical_support.pdf
% for instructions on fixing this problem.

% Note: you may use either harvard or natbib (but not both) to provide a wider
% variety of citation commands than latex supports natively. See below.

% Uncomment the next line to use the natbib package with bibtex 
%\usepackage{natbib}

% Uncomment the next line to use the harvard package with bibtex
%\usepackage[abbr]{harvard}

% This command determines the leading (vertical space between lines) in draft mode
% with 1.5 corresponding to "double" spacing.
\draftSpacing{1.5}

\begin{document}

\title{Gerrymandering in the Laboratory: Experimental Design Section}
%\shortTitle{Short title for running head}
\author{An,  Anderson, and Deck\thanks{Surname1: affiliation1, address1, email1. 
Surname2: affiliation2, address2, email2. Surname3: affiliation3, address3, email3. Acknowledgements}}
%\date{\today}
%\pubMonth{Month}
%\pubYear{Year}
%\pubVolume{Vol}
%\pubIssue{Issue}
%\JEL{}
%\Keywords{}

\begin{abstract}
An experimental design section draft.
\end{abstract}


\maketitle


\section{Experimental Design}
\label{section:experimental_design}

Using the Zurich Toolbox for Ready-made Economic Experiments (Fischbacher, 2007), or Ztree, and The TIDE Lab at The University of Alabama, we develop a setting in which subjects compete against one another across five differently configured maps for a prize of 80 lab dollars, or 20 USD. To avoid experimenter demand effects we use normative language and phrase the competition as an internal sales competition. Each round subjects are randomly matched with an opponent; one subject is Player A and the other is Player B. Subjects complete a series of training exercises in order to ensure they understand how their efforts, or bids, relate to their potential payoffs and how it is they win any given zone, district, or map. A map consists of three districts that are defined by their color: dark gray, light gray, and white. Each district contains three zones. The training exercises walk through these different concepts, testing subjects on their understanding of each and providing them an opportunity to practice, but without giving away the maps on which subjects will actually compete. Player A and Player B both have three guaranteed zones in any given map. The configurations are displayed in Figure \ref{fig:maps}.
\begin{figure}[h]
\centering
\includegraphics[scale=0.4]{maps.png}
\caption{From left to right, maps $Gerry_B$, $Symm_{1,1}$, $Symm_{1,3}$, $Symm_{3,1}$, and $Gerry_A$}
\label{fig:maps}
\end{figure}
For maps $Gerry_A$ and $Gerry_B$ the arrangement of the six pre-determined zones provides an advantage to Player B and Player A, respectively. Any zone that is not pre-determined can be thought of as being open. A subject wins an open zone with probability $\frac{e_{im}}{e_{im}+e_{jm}}$ where $e_{im}$ is the effort/bid chosen by player $i \in \{A,B\}$ in map $m \in \{Gerry_B, Symm_{1,1}, Symm_{1,3}, Symm_{3,1}, Gerry_A\}$ for $i \not= j$. To win a district a subject must win 2/3 of the zones in that district. Subjects select bids for each district, not each zone, so the effort chosen for a district is the effort used when determining the probability a participant wins a singular zone within the relevant district. Figure \ref{fig:map_bids} displays the bidding process subjects complete in each round. 
\begin{figure}[h]
\centering
\includegraphics[scale=0.2]{map_bids.jpg}
\caption{Bidding process for each Stage}
\label{fig:map_bids}
\end{figure}
To win a map a subject must win 2/3 the districts of that map. Subjects are paid for the outcome of a randomly chosen round with each round equally likely. Total bids for any given map may not exceed the contest prize of 80 lab dollars to prevent negative payoffs.
After each round in Stage 1 (the first 10 potentially paid rounds) a map is chosen at random and the outcome of that map determines the subjects' potential payoff for that round. This ensures subjects have proper incentive to make thoughtful decisions for each map in each round. For Stage 2, the three rounds following Stage 1, subjects are also asked for their map preference prior to placing their effort bids in each map. In other words, subjects identify on which map they would like to compete, enabling us to identify whether participants gerrymander when given the opportunity. In Stage 2 a map is chosen at random from the map selections of paired subjects in order to maintain incentive for thoughtful decisions on every map. For the next and final round, Stage 3, subjects are told that their player role, Player A or Player B, is not yet determined, but that they must pick which map they would like to compete on nonetheless. After subjects make their map choice they are assigned a player role and must place bids with this knowledge. The stage then proceeds as Stages 1 and 2. For each Stage, after map selections and bids are made and both subjects submit their choices, the results of the contest for each map are displayed with the randomly chosen map highlighted, showcasing which map will determine the subjects' earnings should that round be selected as the paid round\footnote{The paid round was chosen via a die roll in front of all participants}. The information shown to subjects includes their bids for every district in every map, their opponent's bids in every district in every map, their probability of winning any given district, their probability of wining the map, and their payoff for each map. This informational screen is shown in Figure \ref{fig:info_screen}.
\begin{figure}[h]
\centering
\includegraphics[scale=0.2]{informational_screen.jpg}
\caption{Information presented to subjects at the end of each round}
\label{fig:info_screen}
\end{figure}
\end{document}


