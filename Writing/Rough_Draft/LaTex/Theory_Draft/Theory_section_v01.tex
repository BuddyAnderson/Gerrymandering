% AER-Article.tex for AEA last revised 22 June 2011
\documentclass[AER]{AEA}
\usepackage{caption}
\usepackage{subcaption}
\usepackage{graphicx}
\graphicspath{ {images/} }
% The mathtime package uses a Times font instead of Computer Modern.
% Uncomment the line below if you wish to use the mathtime package:
%\usepackage[cmbold]{mathtime}
% Note that miktex, by default, configures the mathtime package to use commercial fonts
% which you may not have. If you would like to use mathtime but you are seeing error
% messages about missing fonts (mtex.pfb, mtsy.pfb, or rmtmi.pfb) then please see
% the technical support document at http://www.aeaweb.org/templates/technical_support.pdf
% for instructions on fixing this problem.

% Note: you may use either harvard or natbib (but not both) to provide a wider
% variety of citation commands than latex supports natively. See below.

% Uncomment the next line to use the natbib package with bibtex 
%\usepackage{natbib}

% Uncomment the next line to use the harvard package with bibtex
%\usepackage[abbr]{harvard}

% This command determines the leading (vertical space between lines) in draft mode
% with 1.5 corresponding to "double" spacing.
\draftSpacing{1.5}

\begin{document}

\title{Gerrymandering in the Laboratory: Theory Section}
%\shortTitle{Short title for running head}
\author{An,  Anderson, and Deck\thanks{Surname1: affiliation1, address1, email1. 
Surname2: affiliation2, address2, email2. Surname3: affiliation3, address3, email3. Acknowledgements}}
%\date{\today}
%\pubMonth{Month}
%\pubYear{Year}
%\pubVolume{Vol}
%\pubIssue{Issue}
%\JEL{}
%\Keywords{}

\begin{abstract}
A Theory section draft.
\end{abstract}


\maketitle


\section{Theory}
\label{section:Theory}

Configuring voting districts plays an important role in determining the likelihood of a party's candidate winning in many U.S. political competitions. When choosing where and how to campaign, political parties consider the composition of voting districts. If a district leans relatively towards one political party, then an opposing political party may opt not to campaign there and instead allocate their resources to more competitive districts. 

The use of five separate maps, $Gerry_A, Gerry_B, Symm_{1,1}, Symm_{1,3}$ and $Symm_{3,1}$, helps us study participant decisions of Player A and Player B on different districting maps of partisan and undecided voters. For each map a subject's probability of winning is calculated using Tullock contests and tied directly to their bids and the bids of their opponent. The functional form of these probabilities, as functions of each player's bid, differ depending on which map configuration is considered. Using these probabilities we construct a simple expected payoff maximization problem for each map and calculate the optimal bidding strategies as functions of value, $v$. In Appendix, we show the theoretical predictions of optimal total bidding strategies and expected payoff of each player for each map. The results are displayed in Table \ref{Tab:theory_predictions}.
\begin{table}[ht]
\caption{Optimal Bids and Expected Payoff Results} % title of Table
\centering % used for centering table
\begin{tabular}{c c c c c c} % centered columns (4 columns)
\hline\hline %inserts double horizontal lines
 & $Gerry_A$ & $Gerry_B$ & $Symm_{1,1}$ & $Symm_{1,3}$ & $Symm_{3,1}$ \\ [0.5ex] % inserts table heading
\hline \\[-1.8ex]
\vspace{0.2cm}
Opt.tot.bid & $\frac{1}{4}v$ & $\frac{1}{4}v$ & $\frac{1}{4}v$ & $\frac{3}{8}v$ & $\frac{3}{8}v$ \\ % inserting body of the table
\vspace{0.2cm}
$Exp.payoff_A$ & 0 & $\frac{1}{2}v$ & $\frac{1}{4}v$ & $\frac{1}{8}v$ & $\frac{1}{8}v$ \\
$Exp.payoff_B$ & $\frac{1}{2}v$ & 0 & $\frac{1}{4}v$ & $\frac{1}{8}v$ & $\frac{1}{8}v$ \\ [1ex] % [1ex] adds vertical space
\hline %inserts single line
\end{tabular}
\label{Tab:theory_predictions} % is used to refer this table in the text
\end{table}
Notice that equilibrium bids at the map level are equivalent regardless of player types. That is, within $Gerry_i$ or in $Symm_{j,k}$, both players are expected to bid the same amount of effort. As we can see from the Table \ref{table:opt}, equilibrium bids vary across maps $Gerry_i$ and $Symm_{1,1}$ being socially efficient at equilibrium bids of $\frac{1}{4}v$ while $Symm_{1,3}$ and $Symm_{3,1}$ are socially inferior with equilibrium bids of $\frac{3}{8}v$. Therefore we expect to observe no effect of assignment of roles on participants decisions. The last two rows of the Table \ref{table:opt} show expected payoff of each player across maps. Although participants could potentially earn the highest expected payoff when they compete on advantaged maps $Gerry_i$, disadvantaged player could also potentially earn payoff of zero. This suggests participants who do not know their player role should wish to compete on $Symm_{1,1}$ which gives the greatest payoff of $\frac{1}{4}v$ in order to maximize their expected payoff.

The design of maps $Gerry_A, Gerry_B, Symm_{1,1}$, and $Symm_{1,3}$ are such that only the White district actually affects the probability a player wins the contest, incentivizing players to bid zero in two other districts. Maps $Symm_{1,1}, Symm{1,3}$ and $Symm_{3,1}$ are symmetric for both players. In map $Symm_{3,1}$, all three districts are competitive thus the optimal strategy is to bid $\frac{1}{8}v$ in each district, totalling $\frac{3}{8}v$. Another feature of our design is the advantaged maps $Gerry_i$, where Player i already occupies one of three zones in the White district, giving one of the Player a higher probability of winning. Interestingly the optimal bids are not player dependent for these advantaged maps. 
  
Using these features of our design, we compare aforementioned theoretical predictions to actual participant decisions throughout the experiment. In the first part of map selection stage, in order to investigate if participants prefer maps that are personally advantageous, we ask participants to choose the map they would like to compete knowing their roles. Under the assumption that participants are only interested in winning the contest they should pick the map that provides them the greatest chance of winning. In other words, during Map selection stage, Player A should pick $Gerry_A$ and Player B should choose $Gerry_B$. In the second part of map selection stage, we further investigate if participants prefer fair maps when they do not know their roles beforehand. We expect both players to avoid advantaged maps and choose any one of the symmetric maps, $Symm_{1,1}, Symm_{1,3}$ and $Symm_{3,1}$.  
\end{document}