% AER-Article.tex for AEA last revised 22 June 2011
\documentclass[AER]{AEA}
\usepackage{caption}
\usepackage{subcaption}
\usepackage{graphicx}
\graphicspath{ {images/} }
\draftSpacing{1.5}

\begin{document}

\title{Gerrymandering in the Laboratory}
%\shortTitle{Draft}
\author{An,  Anderson, and Deck}

\begin{abstract}
A rough draft.
\end{abstract}


\maketitle

%%%%%%%%%%%%%%%%%%%%%%%%%%%%%%%%%%%%%%%%%%%%%%%%%%%%%%%%%%%%%%%%%%%%%%%%%%

\section{Experimental Design}
\label{section:experimental_design}

Using the Zurich Toolbox for Ready-made Economic Experiments (Fischbacher, 2007), or Ztree, and The TIDE Lab at The University of Alabama, we develop a setting in which subjects compete against one another across five differently configured maps for a prize of 80 lab dollars, or 20 USD. To avoid experimenter demand effects we use normative language and phrase the competition as an internal sales competition. Each round subjects are randomly matched with an opponent; one subject is Player A and the other is Player B. Subjects complete a series of training exercises in order to ensure they understand how their efforts, or bids, relate to their potential payoffs and how it is they win any given zone, district, or map. A map consists of three districts that are defined by their color: dark gray, light gray, and white. Each district contains three zones. Training exercises walk through these different concepts, testing subjects on their understanding of each and providing them an opportunity to practice, but without giving away the maps (below) on which subjects will actually compete. Player A and Player B both have three guaranteed zones in any given map. The configurations are displayed in Figure \ref{fig:maps}.
\begin{figure}[h]
\centering
\includegraphics[scale=0.4]{maps.png}
\caption{From left to right, maps $Gerry_B$, $Symm_{1,1}$, $Symm_{1,3}$, $Symm_{3,1}$, and $Gerry_A$}
\label{fig:maps}
\end{figure}
For maps $Gerry_A$ and $Gerry_B$ the arrangement of the six pre-determined zones provides an advantage to Player B and Player A, respectively. Such advantages can be thought of as gerrymandering in favor of the player to whom the map grants an advantage. All other maps, as their labels suggests, are symmetric and thus offer no advantage to either player. Any zone that is not pre-determined can be thought of as being open. A subject wins an open zone with probability $\frac{e_{im}}{e_{im}+e_{jm}}$ where $e_{im}$ is the effort/bid chosen by player $i \in \{A,B\}$ in map $m \in \{Gerry_B, Symm_{1,1}, Symm_{1,3}, Symm_{3,1}, Gerry_A\}$ for $i \not= j$. Competitions for each zone are therefore of the standard Tullock variety. To win a district a subject must win 2/3 of the zones in that district. Subjects select bids for each district, not each zone, so the effort chosen for a district is the effort used when determining the probability a participant wins a singular zone within the relevant district. Figure \ref{fig:map_bids} displays the bidding process subjects complete in each round.
\begin{figure}[h]
\centering
\includegraphics[scale=0.2]{map_bids.jpg}
\caption{Bidding process for each Stage}
\label{fig:map_bids}
\end{figure}
To win a map a subject must win 2/3 the districts of that map. Subjects are paid for the outcome of a randomly chosen round with each round equally likely. Total bids for any given map may not exceed the contest prize of 80 lab dollars to prevent negative payoffs.
Participants make decisions in 28 rounds spanning three stages. After each round in Stage 1, the first 10 potentially paid rounds, a map is chosen at random and the outcome of that map determines the subjects' potential payoff for that round. This ensures subjects have proper incentive to make thoughtful decisions for each map in each round. For Stage 2, or the three rounds following Stage 1, subjects are also asked for their map preference prior to placing their effort bids in each map. In other words, subjects identify on which map they would like to compete, enabling us to identify whether participants gerrymander when given the opportunity. In Stage 2 a map is chosen at random from the map selections of paired subjects in order to maintain incentive for thoughtful decisions on every map. For the next and final round, Stage 3, subjects are told that their player role, Player A or Player B, is not yet determined, but that they must pick which map they would like to compete on nonetheless. After subjects make their map choice they are assigned a player role and must place bids with this knowledge. The stage then proceeds as Stages 1 and 2. For each Stage, after map selections and bids are made and both subjects submit their choices, the results of the contest for each map are displayed with the randomly chosen map highlighted, showcasing which map will determine the subjects' earnings should that round be selected as the paid round\footnote{The paid rounds were chosen via a die roll in front of all participants}. The information shown to subjects includes their bids for every district in every map, their opponent's bids in every district in every map, their probability of winning any given district, their probability of wining the map, and their payoff for each map. This informational screen is shown in Figure \ref{fig:info_screen}.
\begin{figure}[h]
\centering
\includegraphics[scale=0.2]{informational_screen.jpg}
\caption{Information presented to subjects at the end of each round}
\label{fig:info_screen}
\end{figure}
%%%%%%%%%%%%%%%%%%%%%%%%%%%%%%%%%%%%%%%%%%%%%%%%%%%%%%%%%%%%%%%%%%%%%%%%%%

\section{Theory}
\label{section:Theory}

In this section we define optimal bidding behavior within each of the five distinct maps. As mentioned in Section \ref{section:experimental_design}, a player wins any given map by successfully winning 2/3 of the districts within that map. Using the Tullock style probability success function also specified in Section \ref{section:experimental_design}, we determine the probability a participant wins a given map as a function of each players' bid. Solving for Nash Equilibrium bids using the subsequent expected profit function yields the theoretical predictions available in Table \ref{Tab:theory_predictions}\footnote{See Appendix for details}. Interestingly, we find that Player A and Player B maximize their expected profits by bidding equivalent amounts within any given map, including gerrymandered maps. We therefore report only one optimal bid for each map as both players bid this amount in equilibrium, but expected profits may differ depending on the relevant map's configuration.

\begin{table}[ht]
\caption{Optimal Bids and Expected Payoff Results} 
\centering 
\begin{tabular}{c c c c c c} 
\hline\hline 
 & $Gerry_A$ & $Gerry_B$ & $Symm_{1,1}$ & $Symm_{1,3}$ & $Symm_{3,1}$ \\ [0.5ex]
\hline \\[-1.8ex]
\vspace{0.2cm}
Opt.tot.bid & $\frac{1}{4}v$ & $\frac{1}{4}v$ & $\frac{1}{4}v$ & $\frac{3}{8}v$ & $\frac{3}{8}v$ \\
\vspace{0.2cm}
$Exp.payoff_A$ & 0 & $\frac{1}{2}v$ & $\frac{1}{4}v$ & $\frac{1}{8}v$ & $\frac{1}{8}v$ \\
$Exp.payoff_B$ & $\frac{1}{2}v$ & 0 & $\frac{1}{4}v$ & $\frac{1}{8}v$ & $\frac{1}{8}v$ \\ [1ex]
\hline 
\end{tabular}
\label{Tab:theory_predictions}
\end{table}

Equilibrium bids vary across maps, with $Gerry_A$, $Gerry_B$, and $Symm_{1,1}$ being socially efficient at equilibrium bids of $\frac{1}{4}v$, while $Symm_{1,3}$ and $Symm_{3,1}$ are socially inferior with equilibrium bids of $\frac{3}{8}v$\footnote{Social efficiency here implies the minimum sum of equilibrium bids within a particular map or maps}. The last two rows of Table \ref{Tab:theory_predictions} show expected payoff for each player across maps. If a participant is seeking to maximize their profit, upon entering Stage 2 they ought to select whichever map will provide them with the highest expected profit. That is, we theorize Player A selects $Gerry_A$ and Player B selects $Gerry_B$. During Stage 3, however, a player selects a map without knowing their role. If participants choose a gerrymandered map then their expected profit is $\frac{1}{4}v$ as half the time they will be the advantaged player and the other half disadvantaged. If players choose $Symm_{1,1}$ they face the same expected profit, but only under risk neutrality are these map choices equivalent in terms of player preference. The design of maps $Gerry_A, Gerry_B, Symm_{1,1}$, and $Symm_{1,3}$ are such that only the White district actually affects the probability a player wins the contest, incentivizing players to bid zero in the dark gray and light gray districts. In map $Symm_{3,1}$, all three districts are competitive and the optimal strategy is to bid $\frac{1}{8}v$ in each district, hence the total equilibrium bid of $\frac{3}{8}v$.

%%%%%%%%%%%%%%%%%%%%%%%%%%%%%%%%%%%%%%%%%%%%%%%%%%%%%%%%%%%%%%%%%%%%%%%%%%

\section{Results}
\label{section:results}

\subsection{Stage 1}
\label{subsection:Stage_1}

To introduce our analysis of Stage 1, consider this brief qualitative summary of bidding behavior throughout the study. Figure \ref{fig:full_bidding_time_series} highlights two key features of bidding behavior. 
\begin{figure}[h]
\centering
\includegraphics[scale=0.5]{full_bidding_ts_one_to_ten}
\caption{Average bidding in each session and period for competitive districts}
\label{fig:full_bidding_time_series}
\end{figure}
The first is that salience is achieved. Participants bid much higher in competitive districts than noncompetitive districts with average bids being higher than theoretical predictions. The second feature of the time series plots is that a slight learning effect is present in a few of the noncompetitive districts. In $Gerry_B$ ELG, $Symm_{1,1}$ ELG, and $Symm_{1,3}$ ELG bidding during the first two periods is relatively high and falls as the experiment continues. Another feature of Figure \ref{fig:full_bidding_time_series} is that the panels displaying average bids in districts of $Symm_{3,1}$ look as expected relative to all other panels. That is, bidding in each of the three competitive districts of $Symm_{3,1}$ is less than bids in the singularly competitive, white districts of the other four maps. However, our theoretical prediction of equal bids in each district of $Symm_{3,1}$ appears to be slightly lower than the realized average bids.

Though the averages appear somewhat similar across $Symm_{3,1}$ districts, it is possible that participants only bid in one or two districts rather than bidding in all three. Table \ref{Tab:pct_bid_symm31} and Figure \ref{fig:unadjusted_spread} offer a more granular window into $Symm_{3,1}$ bidding. We see that 72\% of participants partially follow the equilibrium bidding strategy and bid in all three districts. Figure \ref{fig:unadjusted_spread} displays the difference between the maximum and minimum bids for participants who bid in only two districts and those who bid in all three. The inclusion of Figure \ref{fig:unadjusted_spread} helps emphasize that participants mostly bid equal amounts in the districts for which they compete, especially when bidding in only two districts.
\begin{table}
\begin{center}
 \begin{tabular}{|c c c c|} 
 \hline
 \multicolumn{4}{|c|}{ Percent of Participants Bidding in $Symm_{3,1}$} \\
 \hline
 Zero Districts & One District & Two Districts & Three Districts \\ [0.5ex] 
 \hline
 8 & 3 & 17 & 72 \\ 
 \hline
\end{tabular}
\caption{Bidding behavior in $Symm_{3,1}$ with multiple, competitive districts}
\label{Tab:pct_bid_symm31}
\end{center}
\end{table}

\begin{figure}[h]
\centering
\includegraphics[scale=0.3]{unadjusted_spread}
\caption{Difference between maximum and minimum positive bids}
\label{fig:unadjusted_spread}
\end{figure}

As Section \{insert theory section here\} explains, there should be no distinction between bids of Player A and that of Player B. Figure \ref{fig:cdfs} allows us to visualize the influence, if any, of player role. A fairly clear difference between advantaged and disadvantaged players exists, but this only implies bidding varies in gerrymandered maps, not across Player A and Player B participants. A K-S test confirms that the distribution from which advantaged players select their bids is different than that of disadvantaged players. Aligning with our theoretical predictions, it is more difficult to distinguish an effect of player role on bids in Figure \ref{fig:cdfs}. 
\begin{figure}
    \centering
    \begin{subfigure}[t]{0.45\textwidth}
        \centering
        \includegraphics[scale = .35]{symm_1_1_cdf.jpeg} 
        \caption{$Symm_{1,1}$} \label{fig:symm_1_1_cdf}
    \end{subfigure}
    \hfill
    \begin{subfigure}[t]{0.45\textwidth}
        \centering
        \includegraphics[scale = .35]{symm_1_3_cdf.jpeg}
        \caption{$Symm_{1,3}$} \label{fig:symm_1_3_cdf}
    \end{subfigure}

    
    \begin{subfigure}[t]{0.45\textwidth}
        \centering
        \includegraphics[scale = .35]{symm_3_1_cdf.jpeg} 
        \caption{$Symm_{3,1}$} \label{fig:symm_3_1_cdf}
    \end{subfigure}
    \hfill
    \begin{subfigure}[t]{0.45\textwidth}
        \centering
        \includegraphics[scale = .35]{adv_vs_disadv_cdf.jpeg} 
        \caption{$Gerrymandered$} \label{fig:adv_vs_disadv_cdf}
    \end{subfigure}
    \caption{Cumulative distribution functions of bidding in each map by player role}
\label{fig:cdfs}
\end{figure}

To better determine whether a player role effect exists, we begin our quantitative analysis with a linear model including player role effect, map effect, and the interaction of player role and each map. Specifically,
\begin{multline}\label{model_1}
Bid = \alpha + \beta_1 Player_B + \beta_2 Gerry_B + \beta_3 Gerry_B Player_B + \beta_4 Symm_{1,3} + \beta_5 Symm_{1,3} Player_B \\ + \beta_6 Symm_{3,1} + \beta_7 Symm_{3,1} Player_B + \beta_8 Gerry_A + \beta_9 Gerry_A Player_B + \varepsilon
\end{multline}
where $\alpha$ captures bidding of Player A on $Symm_{1,1}$ and $\varepsilon \sim N(0, \sigma^2)$. The choice of $Symm_{1,1}$ as the baseline map allows us to interpret changes in bidding relative to a socially optimal, non-gerrymandered map. If participants follow the equilibrium bidding strategies, there should be no effect of moving from $Symm_{1,1}$ to $Gerry_{A}$ or $Gerry_{B}$. From our theoretical predictions, we expect positive changes when moving from $Symm_{1,1}$ to either $Symm_{1,3}$ or $Symm_{3,1}$. Coefficient estimates presented in Table \{insert first regression\} support our expectation for player role in that there is no significant effect of moving from bidding on $Symm_{1,1}$ as Player A to bidding as Player B in the same map. 
\begin{table}[!htbp] \centering 
  \caption{Model 1 Regression Results} 
  \label{Tab:regression_1} 
\begin{tabular}{@{\extracolsep{5pt}}lcc} 
\\[-1.8ex]\hline 
\hline \\[-1.8ex] 
 & \multicolumn{2}{c}{\textit{Dependent variable:}} \\ 
\cline{2-3} 
\\[-1.8ex] & \multicolumn{2}{c}{Effort} \\ 
 & w/out learning & w/ learning \\ 
\\[-1.8ex] & (1) & (2)\\ 
\hline \\[-1.8ex] 
 Player\_B & 1.219 (2.345) & 1.219 (2.345) \\ 
  Gerry\_B & $-$4.325$^{*}$ (2.345) & $-$4.325$^{*}$ (2.345) \\ 
  Symm\_1\_3 & 0.575 (2.345) & 0.575 (2.345) \\ 
  Symm\_3\_1 & 7.431$^{***}$ (2.345) & 7.431$^{***}$ (2.345) \\ 
  Gerry\_A & $-$1.331 (2.345) & $-$1.331 (2.345) \\ 
  Player\_B:Gerry\_B & 2.563 (3.317) & 2.563 (3.317) \\ 
  Player\_B:Symm\_1\_3 & 1.194 (3.317) & 1.194 (3.317) \\ 
  Player\_B:Symm\_3\_1 & $-$1.531 (3.317) & $-$1.531 (3.317) \\ 
  Player\_B:Gerry\_A & $-$3.194 (3.317) & $-$3.194 (3.317) \\ 
  Constant & 45.656$^{***}$ (1.658) & 45.656$^{***}$ (1.658) \\ 
 \hline \\[-1.8ex] 
Observations & 1,600 & 1,600 \\ 
R$^{2}$ & 0.031 & 0.031 \\ 
Adjusted R$^{2}$ & 0.025 & 0.025 \\ 
Residual Std. Error (df = 1590) & 20.978 & 20.978 \\ 
F Statistic (df = 9; 1590) & 5.603$^{***}$ & 5.603$^{***}$ \\ 
\hline 
\hline \\[-1.8ex] 
\textit{Note:}  & \multicolumn{2}{r}{$^{*}$p$<$0.1; $^{**}$p$<$0.05; $^{***}$p$<$0.01} \\ 
\end{tabular} 
\end{table} 
Linear hypothesis testing of each possible combination of Player B effects and map level effects provides further support that player role does not impact bidding behavior. Unlike our prediction, we do not find statistically significant support for increased bidding in $Symm_{1,3}$ relative to $Symm_{1,1}$. However, the coefficient on $Symm_{3,1}$ is both positive and significant ($\hat{\beta_6} = 7.33$, $\mathit{p\mbox{-}value} = 0.000$). Though Table \ref{Tab:regression_1} provides estimates for the effects of gerrymandered maps, we have shown player role does not change bidding behavior. Therefore, interpreting the effects of gerrymandered maps is best done in the context of an advantaged or disadvantaged map. That is, we estimate
\begin{multline}\label{model_2}
Bid = \alpha + \beta_1 Advantage + \beta_2 Disadvantage + \beta_3 Symm_{1,3} + \beta_4 Symm_{3,1} + \beta_5 Stage^{II} \\ + \beta_6 Advantage Stage^{II} + \beta_7 Disadvantage Stage^{II} + \beta_8 Symm_{1,3} Stage^{II} + \beta_9 Symm_{3,1} Stage^{II} + \varepsilon
\end{multline} 
where $Stage^{II}$ is an indicator variable which captures the effect of map selection, discussed further in section \ref{subsection:Stage_2}. These estimates, reported in Table \ref{Tab:regression_2}, rely on all data up to and including the last period of Stage 2. 
\begin{table}[!htbp] \centering 
  \caption{Model 2 Regression Results} 
  \label{Tab:regression_2} 
\begin{tabular}{@{\extracolsep{5pt}}lcc} 
\\[-1.8ex]\hline 
\hline \\[-1.8ex] 
 & \multicolumn{2}{c}{\textit{Dependent variable:}} \\ 
\cline{2-3} 
\\[-1.8ex] & \multicolumn{2}{c}{Effort} \\ 
 & w/out learning & w/ learning \\ 
\\[-1.8ex] & (1) & (2)\\ 
\hline \\[-1.8ex] 
 Adv & $-$1.470 (1.219) & $-$1.547 (1.705) \\ 
  Disadv & $-$3.656$^{***}$ (1.219) & $-$4.425$^{***}$ (1.705) \\ 
  Symm\_1\_3 & 1.417 (1.219) & 1.172 (1.705) \\ 
  Symm\_3\_1 & 7.189$^{***}$ (1.219) & 6.666$^{***}$ (1.705) \\ 
  Stage\_2\_indicator & $-$5.405$^{***}$ (1.795) & $-$4.271$^{**}$ (1.969) \\ 
  Adv:Stage\_2\_indicator & 2.215 (2.538) & 2.292 (2.784) \\ 
  Disadv:Stage\_2\_indicator & 0.224 (2.538) & 0.993 (2.784) \\ 
  Symm\_1\_3:Stage\_2\_indicator & 0.822 (2.538) & 1.068 (2.784) \\ 
  Symm\_3\_1:Stage\_2\_indicator & $-$2.408 (2.538) & $-$1.884 (2.784) \\ 
  Constant & 47.400$^{***}$ (0.862) & 46.266$^{***}$ (1.206) \\ 
 \hline \\[-1.8ex] 
Observations & 4,160 & 2,560 \\ 
R$^{2}$ & 0.034 & 0.030 \\ 
Adjusted R$^{2}$ & 0.032 & 0.027 \\ 
Residual Std. Error & 21.813 (df = 4150) & 21.568 (df = 2550) \\ 
F Statistic & 16.343$^{***}$ (df = 9; 4150) & 8.844$^{***}$ (df = 9; 2550) \\ 
\hline 
\hline \\[-1.8ex] 
\textit{Note:}  & \multicolumn{2}{r}{$^{*}$p$<$0.1; $^{**}$p$<$0.05; $^{***}$p$<$0.01} \\ 
\end{tabular} 
\end{table}
In this setting, the coefficients of Advantage and Disadvantage both indicate that bids for any gerrymandered map are less than those in $Symm_{1,1}$, but only in the case of a Disadvantaged map is the coefficient statistically significant ($\hat{\beta_2} = -3.66$,$\mathit{p\mbox{-}value} = 0.0027$). From this regression we again report positive changes in bid amounts for $Symm_{1,3}$ and $Symm_{3,1}$ and as with the previous regression, only the map with three open districts has a statistically significant coefficient ($\hat{\beta_4} = 7.19$, $\mathit{p\mbox{-}value} = 0.000$). To explain our inclusion of a Stage 2 indicator in our section devoted to Stage 1 we draw attention to the negative coefficient of the indicator covariate itself ($\hat{\beta_5} = -5.41$, $\mathit{p\mbox{-}value} = 0.0026$). Given that there is no apparent reason for participants to change their bidding strategies simply because they are able to select a map, which is not guaranteed to be the map from which payment is determined, this decline in bidding is perplexing. As a possible explanation for this behavioral shift, we model bidding as a function of map and period (or time) to account for learning effects during Stage 1. Table \ref{Tab:regression_3} presents the estimates of the model
\begin{multline}\label{model_3}
Bid = \alpha + \beta_1 Advantage + \beta_2 Disadvantage + \beta_3 Symm_{1,3} + \beta_4 Symm_{3,1} + \beta_5 Period
 \\ + \beta_6 Advantage Period + \beta_7 Disadvantage Period + \beta_8 Symm_{1,3} Period + \beta_9 Symm_{3,1}Period + \varepsilon
\end{multline}
for which we run a joint linear hypothesis test on each effect of $Period$. The estimate of the $Period$ coefficient is negative and statistically significant ($\hat{\beta_5} = -0.67$, $\mathit{p\mbox{-}value} = 0.0235$), indicating that as participants advance through Stage 1 they reduce their bids, albeit by fairly small amounts. As a robustness check, we evaluate each of the previous models under the assumption that learning happens during the first half of Stage 1 and, after five periods of the same environment and institution, participants have converge on their individual strategies. Tables \ref{Tab:regression_1}, \ref{Tab:regression_2}, and \ref{Tab:regression_3} report the estimates for the previous models under the constraint that periods 15 through 19 are not included in the second column of each table. We draw the same conclusion, that player role does not effect bidding, with the abbreviated data. The same is true for model \ref{model_2} under the abbreviated data. We again observe negative coefficients for the gerrymandered maps with only the Disadvantaged covariate having a statistically significant result ($\hat{\beta_2} = -4.43$,$\mathit{p\mbox{-}value} = 0.0095$). The Stage 2 indicator also maintains a significant, negative effect ($\hat{\beta_5} = -4.27$, $\mathit{p\mbox{-}value} = 0.0302$). The impact of the $Period$ estimate for model (\ref{model_3}) is no longer significant, but remains negative ($\hat{\beta_5} = -0.99$, $\mathit{p\mbox{-}value} = 0.2333$) as shown in Table \ref{Tab:regression_3}.
\begin{table}[!htbp] \centering 
  \caption{Model 3 Regression Results} 
  \label{Tab:regression_3} 
\begin{tabular}{@{\extracolsep{5pt}}lcc} 
\\[-1.8ex]\hline 
\hline \\[-1.8ex] 
 & \multicolumn{2}{c}{\textit{Dependent variable:}} \\ 
\cline{2-3} 
\\[-1.8ex] & \multicolumn{2}{c}{Effort} \\ 
 & w/out learning & w/ learning \\ 
\\[-1.8ex] & (1) & (2)\\ 
\hline \\[-1.8ex] 
 Adv & $-$1.299 (8.253) & $-$9.522 (25.821) \\ 
  Disadv & $-$1.595 (8.253) & $-$0.988 (25.821) \\ 
  Symm\_1\_3 & $-$1.126 (8.253) & $-$10.516 (25.821) \\ 
  Symm\_3\_1 & 8.598 (8.253) & 6.116 (25.821) \\ 
  Period & $-$0.671$^{**}$ (0.296) & $-$0.988 (0.828) \\ 
  Adv:Period & $-$0.009 (0.419) & 0.363 (1.171) \\ 
  Disadv:Period & $-$0.106 (0.419) & $-$0.156 (1.171) \\ 
  Symm\_1\_3:Period & 0.130 (0.419) & 0.531 (1.171) \\ 
  Symm\_3\_1:Period & $-$0.072 (0.419) & 0.025 (1.171) \\ 
  Constant & 60.489$^{***}$ (5.836) & 67.991$^{***}$ (18.258) \\ 
 \hline \\[-1.8ex] 
Observations & 3,200 & 1,600 \\ 
R$^{2}$ & 0.036 & 0.033 \\ 
Adjusted R$^{2}$ & 0.033 & 0.028 \\ 
Residual Std. Error & 21.514 (df = 3190) & 20.952 (df = 1590) \\ 
F Statistic & 13.257$^{***}$ (df = 9; 3190) & 6.050$^{***}$ (df = 9; 1590) \\ 
\hline 
\hline \\[-1.8ex] 
\textit{Note:}  & \multicolumn{2}{r}{$^{*}$p$<$0.1; $^{**}$p$<$0.05; $^{***}$p$<$0.01} \\ 
\end{tabular} 
\end{table} 

To close out our analysis of Stage 1 consider the effects we have shown. First, participants over bid relative to equilibrium on every map. On socially inefficient maps bidding does not increase at much as predicted relative to socially efficient maps. Socially inefficient maps are also not equivalent in realized bids, as theory suggests, with $Symm_{3,1}$ extracting higher bids than $Symm_{1,3}$. Secondly, when participants see themselves as advantaged or disadvantaged they bid less, but being disadvantaged augments this downward effect. Third, while learning occurs throughout Stage 1 and dampens bids, the effect is quite small. The fourth and final observation from the preceding results is the effect of Stage 2. When able to influence, to some degree, the map on which they will compete for the prize, bidding declines relative to Stage 1 bids.

\subsection{Stage 2}
\label{subsection:Stage_2}

As mentioned, our theoretical predictions suggest that bidding on any map is not impacted by the ability to select a map on which to compete. The reason for this is that any map \emph{could} be the map for which the competition actually translate to a payout and participants should therefore maintain the same bidding strategies they implemented in Stage 1. In practice this is not the case. Later in this section we offer possible explanations for this deviation from theory, but we must address which maps participants actually select.
\begin{figure}[h]
\centering
\includegraphics[scale=0.5]{map_choice_stage_2.jpeg}
\caption{Modal choice of map for each participant}
\label{fig:map_choice_stage_2}
\end{figure}
We report the modal choice of participants when asked to select a map. The reason for using modal map selection is to identify on which map a participant converges\footnote{For participants that choose three different maps we use their map choice in the last period of Stage 2. If we remove the 11 individuals who did not choose the same map at least twice, then we find that 40 of 53 participants gerrymander, which is a little over 75\%.}. Figure \ref{fig:map_choice_stage_2} illustrates that the majority of participants engage in gerrymandering. We find that 43 of 64 participants, or about 67\%, prefer the gerrymandered map that provides them with an advantage. The high prevalence of gerrymandering is interesting when compared with the responses to a post-experiment questionnaire. Figure \ref{fig:gerry_and_politics} displays histograms of participant responses to the question: "On a scale of 1 to 9, how would you describe your political views with 1 being extremely liberal (i.e. to the left of the Democratic Party), 5 being centrist (i.e. falling between the Democratic Party and the Republican Party), and 9 being extremely conservative (i.e. to the right of the Republican party)." The color identifies participants who responded to a separate question: "Do you support gerrymandering (the manipulation of the boundaries of electoral constituencies to favor one election outcome over another)." Clearly, an overwhelming majority of participants, about 95\%, claim to not support gerrymandering. Of those 61 participants, 41 engage in gerrymandering. That is, 41 of 43 gerrymandering participants claim to disapprove of the practice in which they themselves engage.
\begin{figure}[h]
\centering
\includegraphics[scale=0.5]{gerry_and_politics.jpeg}
\caption{Political leaning and decision to gerrymander}
\label{fig:gerry_and_politics}
\end{figure}

\subsection{Stage 3}
\label{subsection:Stage_3}

Our analysis of Stage 3 is largely qualitative. The additional treatment in Stage 3 is map selection under a veil of ignorance. While theory suggests participants maximize expected earnings by choosing $Symm_{1,1}$ and bidding $20$, Figure \ref{fig:map_choice_stage_3} paints a much different picture. In fact, over half of all participants choose socially inefficient maps when unaware of the role they will have in the competition. Further, just under 30\% of participants select gerrymandered maps. This leads to the question: are subjects picking gerrymandered maps because they believe they are going to be in the same role as in previous periods, or are they willing to take the 50/50 chance of ending up in an advantaged map? To answer this question we depict a "spillover" effect in Figure \ref{fig:spillover_unadjusted} where spillover occurs when a participant gerrymandered in Stage 2 and picks that same map in Stage 3. We report 18\% of participants are impacted by this spillover effect and 9\% of participants are simply choosing a gerrymandered map without having chosen the same one in the pervious stage.

\begin{figure}[!h]
\centering
\includegraphics[scale=0.5]{map_choice_stage_3.jpeg}
\caption{Modal choice of map for each participant in Stage 3}
\label{fig:map_choice_stage_3}
\end{figure}
\begin{figure}[!h]
\centering
\includegraphics[scale=0.5]{spillover_unadjusted.jpeg}
\caption{Map choice in Stage 3 by spillover identifier}
\label{fig:spillover_unadjusted}
\end{figure}

\newpage

Table \ref{Tab:all_districts_zero_avg} provides the percentage of participants who bid zero in any given district, the average bid conditional on not bidding zero, and the unconditional average bid for each district.

\begin{table}
\caption{District Statistics} 
\begin{center}
 \begin{tabular}{|c c c c|} 
% \hline
% \multicolumn{4}{|c|}{District Statistics} \\
 \hline
 Map and District & Percent Bidding Zero & Average Positive Bid & Unconditional Average Bid \\ [0.5ex] 
 \hline
 $Gerrymandered_B$ W & 6 & 40.05 & 37.73 \\
 $Gerrymandered_B$ LG & 85 & 17.45 & 2.59 \\
 $Gerrymandered_B$ DG & 77 & 19.84 & 4.59 \\
  \hline
 $Symm_{1,1}$ W & 3 & 40.35 & 39.35 \\
 $Symm_{1,1}$ LG & 80 & 16.03 & 3.26 \\
 $Symm_{1,1}$ DG & 76 & 19.94 & 4.80 \\
  \hline
 $Symm_{1,3}$ W & 3 & 44.51 & 43.40 \\
 $Symm_{1,3}$ LG & 87 & 14.98 & 1.99 \\
 $Symm_{1,3}$ DG & 84 & 21.10 & 3.43 \\
  \hline
 $Symm_{3,1}$ W & 13 & 20.38 & 17.73 \\
 $Symm_{3,1}$ LG & 12 & 21.56 & 18.94 \\
 $Symm_{3,1}$ DG & 16 & 21.24 & 17.92 \\
  \hline
 $Gerrymandered_A$ W & 6 & 39.95 & 37.57 \\
 $Gerrymandered_A$ LG & 80 & 18.32 & 3.75 \\
 $Gerrymandered_A$ DG & 84 & 20.96 & 3.44 \\
 \hline
\end{tabular}
\label{Tab:all_districts_zero_avg}
\end{center}
\end{table}


Table \ref{Tab:all_districts_zero_avg_stage_1_last} provides the percentage of participants who bid zero in any given district, the average bid conditional on not bidding zero, and the unconditional average bid for each district, but here we only look at the last half of Stage 1 (or periods 6-10)

\begin{table}
\caption{District Statistics: second half of Stage 1} 
\begin{center}
 \begin{tabular}{|c c c c|} 
% \hline
% \multicolumn{4}{|c|}{District Statistics: second half of Stage 1} \\
 \hline
 Map and District & Percent Bidding Zero & Average Positive Bid & Unconditional Average Bid \\ [0.5ex] 
 \hline
 $Gerrymandered_B$ W & 6 & 40.21 & 37.95 \\
 $Gerrymandered_B$ LG & 91 & 17.62 & 1.60 \\
 $Gerrymandered_B$ DG & 83 & 21.40 & 3.68 \\
  \hline
 $Symm_{1,1}$ W & 3 & 41.26 & 40.23 \\
 $Symm_{1,1}$ LG & 85 & 15.00 & 2.25 \\
 $Symm_{1,1}$ DG & 82 & 20.56 & 3.79 \\
  \hline
 $Symm_{1,3}$ W & 2 & 44.37 & 43.40 \\
 $Symm_{1,3}$ LG & 92 & 16.73 & 1.36 \\
 $Symm_{1,3}$ DG & 89 & 23.81 & 2.68 \\
  \hline
 $Symm_{3,1}$ W & 12 & 19.99 & 17.68 \\
 $Symm_{3,1}$ LG & 13 & 21.14 & 18.36 \\
 $Symm_{3,1}$ DG & 18 & 20.63 & 16.89 \\
  \hline
 $Gerrymandered_A$ W & 6 & 39.88 & 37.52 \\
 $Gerrymandered_A$ LG & 84 & 20.50 & 3.20 \\
 $Gerrymandered_A$ DG & 88 & 22.05 & 2.62 \\
 \hline
\end{tabular}
\label{Tab:all_districts_zero_avg_stage_1_last}
\end{center}
\end{table}

Model \ref{Tab:regression_4} presents the results of 
\begin{multline}\label{model_4}
Bid = \alpha + \beta_1 Advantage + \beta_2 Disadvantage + \beta_3 Symm_{1,3} + \beta_4 Symm_{3,1} + \varepsilon
\end{multline}
with subject level fixed effects and standard errors clustered at the session level.
\begin{table}[!h] \centering 
  \caption{Model 4 Regression Results} 
  \label{Tab:regression_4} 
\begin{tabular}{@{\extracolsep{5pt}}lc} 
\\[-1.8ex]\hline 
\hline \\[-1.8ex] 
 & \multicolumn{1}{c}{\textit{Dependent variable:}} \\ 
\cline{2-2} 
\\[-1.8ex] & Effort \\ 
\hline \\[-1.8ex] 
 Adv & $-$1.547 (1.030) \\ 
  Disadv & $-$4.425$^{***}$ (1.030) \\ 
  Symm\_1\_3 & 1.172 (1.030) \\ 
  Symm\_3\_1 & 6.666$^{***}$ (1.030) \\ 
 \hline \\[-1.8ex] 
Observations & 1,600 \\ 
R$^{2}$ & 0.076 \\ 
Adjusted R$^{2}$ & 0.036 \\ 
F Statistic & 31.602$^{***}$ (df = 4; 1532) \\ 
\hline 
\hline \\[-1.8ex] 
\textit{Note:}  & \multicolumn{1}{r}{$^{*}$p$<$0.1; $^{**}$p$<$0.05; $^{***}$p$<$0.01} \\ 
\end{tabular} 
\end{table} 
We are able to reject the null that Adv = Disadv and $Symm_{1,3}$ = $Symm_{3,1}$ with both p-values being less than 0.001.

\newpage

The average [total] bid by an advantaged player on a gerrymandered map in the last 5 periods of stage 1 is 44.7

The average [total] bid by a disadvantaged player on a gerrymandered map in the last 5 periods of stage 1 is 41.8

The percentage of wins by an advantaged player on a gerrymandered map for the entire stage 1 is 55\%

The percentage of wins by an advantaged player on a gerrymandered map for the entire stage 1 is 55\%


\end{document}